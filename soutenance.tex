\documentclass[12pt, aspectratio=169]{beamer}

% --- THEME ET CONFIGURATION [cite: 172, 177] ---
\usetheme{Madrid} 
\usecolortheme{beaver} % Couleurs sobres (Gris/Rouge)
\setbeamerfont{normal text}{size=\large} % Police > 18pt en rendu
\setbeamertemplate{navigation symbols}{} % Pas de symboles de nav
\setbeamertemplate{caption}[numbered]{} % Pas de numéros de figure [cite: 205]

\usepackage[utf8]{inputenc}
\usepackage[T1]{fontenc}
\usepackage[french]{babel}
\usepackage{graphicx}

% --- METADONNEES ---
\title[Projet Théorie des Langages]{Développement d'une bibliothèque de manipulation d'automates}
\subtitle{Retour d'expérience - Module TL2}
\author{Elouan BOITEUX \and Aymeric MARIAUX \and Killian MATHIAS}
\institute[Univ. MLP]{Université Marie et Louis Pasteur \\ Licence 3 Informatique}
\date{}

\begin{document}

\begin{frame}
    \titlepage
    \begin{center}
        \small Tuteur : Julien Bernard
    \end{center}
\end{frame}

\begin{frame}{Sommaire}
    \tableofcontents
    \vfill
    \textit{Durée estimée : 20 minutes}
\end{frame}

\section{Contexte et Mission}

\begin{frame}{1. Contexte du projet}
    \textbf{Elouan Boiteux | Durée : 2 min}
    \begin{itemize}
        \item \textbf{Cadre :} Licence 3 Informatique, Module Théorie des Langages.
        \item \textbf{Objectif pédagogique :}
            \begin{itemize}
                \item Comprendre la création de langage à partir d'automates.
                \item Mettre en œuvre des algorithmes complexes.
            \end{itemize}
        \item \textbf{Notre mission :}
            \begin{itemize}
                \item Implémenter une structure Automaton et ses méthodes.
                \item Public cible de ce rapport : Non-experts.
            \end{itemize}
    \end{itemize}
\end{frame}

\begin{frame}{2. C'est quoi un automate ?}
    \textbf{Aymeric Mariaux| Durée : 2 min}
    \begin{columns}
        \column{0.6\textwidth}
        \begin{itemize}
            \item Une machine abstraite simple.
            \item Composants :
                \begin{itemize}
                    \item Des \textbf{États} (Ronds).
                    \item Des \textbf{Transitions} (Flèches avec lettres).
                \end{itemize}
            \item \textbf{Analogie :} Un distributeur de boissons.
                \item État 0 : Attente pièce.
                \item Action "Insérer 1€" $\rightarrow$ État 1 : Choix.
        \end{itemize}
        \column{0.4\textwidth}
        \centering
        % Remplacer par une image réelle
        \includegraphics[width=\textwidth]{./assets/automate.png} 
        \textit{Exemple d'automate}
    \end{columns}
\end{frame}

\section{Réalisation Technique}

\begin{frame}{3. Contraintes et Outils}
    \textbf{Killian Mathias | Durée : 2 min}
    \begin{itemize}
        \item \textbf{Environnement technique :}
            \begin{itemize}
                \item Langage : C++.
                \item Compilation : CMake.
                \item Tests : Google Test Framework.
            \end{itemize}
        \item \textbf{Les difficultés majeures :}
            \begin{itemize}
                \item Choix de structure libre ("Une mauvaise structure amène une mauvaise implémentation")
                \item Algorithmes complexes
            \end{itemize}
    \end{itemize}
\end{frame}

\begin{frame}{4. Choix de Conception}
    \textbf{Elouan Boiteux | Durée : 3 min}
    \begin{block}{Comment représenter l'automate en mémoire ?}
        Structure possédant un ensemble d'état, de symboles (alphabet) et de transitions
    \end{block}
    \begin{itemize}
        \item \textbf{Pourquoi ?}
            \begin{itemize}
                \item Accès rapide aux états par index.
                \item Structure optimisée pour avoir un parcours performant.
            \end{itemize}
    \end{itemize}
\end{frame}

\begin{frame}{5. Algorithmique}
    \textbf{Aymeric Mariaux | Durée : 3 min}\\
    \begin{itemize}
        \item Développement d'algorithmes complexes
        \item Exemple : Déterminisation\begin{itemize}
            \item \textbf{Le problème :} L'automate possède plusieurs choix pour une lettre à partir d'un état.
            \item \textbf{La solution :} Regrouper les états possibles en un nouvel état.
            \item \textbf{Résultat :} Un automate prévisible et programmable.
        \end{itemize}
    \end{itemize}
    \vspace{1em}
    \centering
\end{frame}

\begin{frame}{6. Validation et Tests}
    \textbf{Killian Mathias | Durée : 3 min}
    \begin{itemize}
        \item \textbf{Stratégie de Test :}
            \begin{itemize}
                \item Écriture des tests \textit{avant} le code.
                \item Vérification des cas particuliers (automate vide).
            \end{itemize}
        \item \textbf{La "Moulinette" :}
            \begin{itemize}
                \item Script d'évaluation se reposant sur un ensemble de tests sur notre implémentation.
                \item Résultat actuel : 100\% de réussite.
            \end{itemize}
    \end{itemize}
\end{frame}

\section{Organisation et Vécu}

\begin{frame}{7. Gestion de Projet et Vécu}
    \textbf{Elouan Boiteux \& Aymeric Mariaux | Durée : 3 min}
    \begin{itemize}
        \item \textbf{Planification :}
            \begin{itemize}
                \item 12 séances de TP de 1h30.
                \item Avance prise dès le premier TP
                \item Travail personnel (soir/weekend).
            \end{itemize}
        \item \textbf{Difficultés rencontrées :}
            \begin{itemize}
                \item Compréhension des itérateurs C++.
                \item Algorithmes de Moore et Hopcroft
            \end{itemize}
        \item \textbf{Réussite :} Code propre et fonctionnel.
    \end{itemize}
\end{frame}

\section{Conclusion}

\begin{frame}{Conclusion et Bilan}
    \textbf{Killian Mathias | Durée : 2 min}
    \begin{columns}
        \column{0.5\textwidth}
        \textbf{Bilan Technique}
        \begin{itemize}
            \item Implémentation opérationnelle.
            \item Bonus réalisés (dotPrint, Minimisation avec Hopcroft, suppression des $\epsilon$-transitions).
        \end{itemize}
        
        \column{0.5\textwidth}
        \textbf{Bilan Personnel}
        \begin{itemize}
            \item Montée en compétence C++.
            \item Rigueur algorithmique.
            \item Optimisation de structure
        \end{itemize}
    \end{columns}
\end{frame}

\begin{frame}
    \centering
    \Huge \textbf{Merci de votre attention}
    
    \vspace{1cm}
    \Large Avez-vous des questions ?
    
    \vspace{1cm}
    \small \textit{Projet réalisé dans le cadre de l'UE TCPP}
\end{frame}

\end{document}